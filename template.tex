%%%%%%%%%%%%%%%%%%%%%%%%%%%%%%%%%%%%%%%%%
% Twenty Seconds Resume/CV
% LaTeX Template
% Version 1.0 (14/7/16)
%
% Original author:
% Carmine Spagnuolo (cspagnuolo@unisa.it) with major modifications by 
% Vel (vel@LaTeXTemplates.com) and Harsh (harsh.gadgil@gmail.com)
%
% License:
% The MIT License (see included LICENSE file)
%
%%%%%%%%%%%%%%%%%%%%%%%%%%%%%%%%%%%%%%%%%

%----------------------------------------------------------------------------------------
%	PACKAGES AND OTHER DOCUMENT CONFIGURATIONS
%----------------------------------------------------------------------------------------

\documentclass[letterpaper]{twentysecondcv} % a4paper for A4

% Command for printing skill overview bubbles
\newcommand\skills{ 

	\smartdiagram[bubble diagram]{
        \textbf{~~~~Theoretical~~~~}\\\textbf{~~Physics~~},
        \textbf{~~Differential~~}\\\textbf{Geometry},
        \textbf{Functional}\\\textbf{Programming},
        \textbf{~~~~~~Machine~~~~~~}\\\textbf{~~Learning~~},
        \textbf{~~~~Quantum~~~~}\\\textbf{Physics}
    }
}

% Programming skill bars
\programming{{C $\textbullet$ C++ $\textbullet$ Nim $\textbullet$ Elixir / .8 }, {Python $\textbullet$ Julia $\textbullet$ Scala $\textbullet$ R $\textbullet$ Chapel / 2}, {Go $\textbullet$ Haskell $\textbullet$ D / 5}, {Rust / 5.9}}

% Projects text
\education{
\textbf{MSc. Candidate, Physics} \\
Specialization: Particle Physics \\
Yonsei University \\
2017 - | Seoul, South Korea

\textbf{BSc., Astronomy} \\
Yonsei University \\
2012 - 2017 | Seoul, South Korea
}

%----------------------------------------------------------------------------------------
%	 PERSONAL INFORMATION
%----------------------------------------------------------------------------------------
% If you don't need one or more of the below, just remove the content leaving the command, e.g. \cvnumberphone{}

\cvname{Tae Geun Kim} % Your name
\cvjobtitle{ Graduate Student } % Job
% title/career

\cvlinkedin{/in/tae-geun-kim}
\cvgithub{github.com/Axect}
%\cvnumberphone{(647) 221 7999} % Phone number
\cvsite{axect.github.io} % Personal website
\cvmail{edeftg@gmail.com} % Email address

%----------------------------------------------------------------------------------------

\begin{document}

\makeprofile % Print the sidebar
 
%----------------------------------------------------------------------------------------
%	 Research Interests
%----------------------------------------------------------------------------------------

\section{Research Interests}

Geometric method in mathematical physics, General relativity, Scientific computation and Machine learning for High energy physics.
 
%----------------------------------------------------------------------------------------
%	 EXPERIENCE
%----------------------------------------------------------------------------------------

\section{Experience}

\begin{twenty} % Environment for a list with descriptions
    \twentyitem
    	{Mar 2017 -}
		{Feb 2018}
        {Web Assistant}
        {\href{http://science.yonsei.ac.kr/}{College of Science, Yonsei Univ.}}
        {}
        {
        {\begin{itemize}
        \item Developed and maintained new web page of College of Science, Yonsei University with flask.  
        \item Main work: Debugging and data uploads 
        \item Tool: Apache, CSS, JS, Flask, MySQL
    \end{itemize}}
        }
        \\
    \twentyitem
    	{Mar 2017 -}
		{June 2017}
        {Teaching Assistant}
        {\href{https://yonsei.ac.kr/}{Yonsei Univ.}}
        {}
        {
        {\begin{itemize}
        \item TA for PHY4205-01 (Mathematical Physics(1))  
        \item Main work: Delivered several lectures for general relativity and Scoring
        \item Adviser: Seodong Shin, Dept. of Physics, Yonsei Univ.
    \end{itemize}}
        }
        \\
    \twentyitem
    	{Mar 2016 -}
		{Feb 2017}
        {Undergraduate Intern}
        {\href{http://nexus.yonsei.ac.kr/}{Yonsei Univ.}}
        {}
        {\begin{itemize}
        \item Studied quantum field theory, general relativity and performed several projects.
        \item Projects: Practice of HEP calculation tools with $\phi^4$ theory
        
        Tools: MG5\_aMC, CalcHep, Mathematica
        
        \item Adviser: Seong Chan Park, Dept. of Physics, Yonsei Univ.
        \end{itemize}}
        
	%\twentyitem{<dates>}{<title>}{<location>}{<description>}
\end{twenty}

%----------------------------------------------------------------------------------------
%	 EDUCATIONAL EXPERIENCE
%----------------------------------------------------------------------------------------
\section{Educational Experience}
\begin{twenty}
    \twentyitem
        {2018}
        {}
        {Schools}
        {}
        {}
        {\begin{itemize}
        \item \textbf{The 9th KIAS CAC Summer School on Scientific computing and Machine learning} (\href{http://www.kias.re.kr/}{KIAS}, 26-29 June)
        \end{itemize}}
    \\
    \twentyitem
        {2017}
        {}
        {Schools}
        {}
        {}
        {\begin{itemize}
        \item \textbf{The first MadAnalysis 5 workshop on LHC recasting in Korea}
        (\href{https://indico.cern.ch/event/637941/overview}{KIAS}, 20-27 August)
        
        \item \textbf{2017 Summer School on Cosmology and Particle Physics}
        
        (\href{https://indico.ibs.re.kr/event/153/}{IBS-CTPU}, 7-11 August)
        \end{itemize}
        }
    \\
    \twentyitem
        {2016}
        {}
        {Schools}
        {}
        {}
        {\begin{itemize}
        \item \textbf{2016 Winter School on Cosmology and Particle Physics}
        
        (\href{https://indico.ibs.re.kr/event/97/}{IBS-CTPU}, 12-23 December)
        
        \item \textbf{KIAS-QUC Winter School on Collider Physics}
        
        (\href{http://home.kias.re.kr/MKG/h/WSCP/?pageNo=2548}{KIAS}, 26-29 December)
        \end{itemize}}
    \\
    \twentyitem
        {2015}
        {}
        {Schools}
        {}
        {}
        {\begin{itemize}
        \item \textbf{Yangpyeong Particle Physics School}
        
        (\href{http://home.kias.re.kr/MKG/h/YPschool2015/?pageNo=1717}{KIAS}, 17-20 December)
        \end{itemize}
        }
    \\
\end{twenty}

\newpage

\makeprofiletwo
%----------------------------------------------------------------------------------------
%	 RESEARCH
%----------------------------------------------------------------------------------------
\section{Programming Projects}
\begin{twenty}
	\twentyitem
    	{Sep 2018 -}
		{Present}
        {Peroxide}
        {\href{https://crates.io/crates/peroxide}{crates}}
        {Rust numeric library with easy syntax}
        {\begin{itemize}
        \item Rust numeric library contains linear algebra, numerical analysis, statistics and machine learning tools with R, MATLAB, Python like macros.
        \item Tools: Rust, Cargo, Travis CI
		\end{itemize}
        }
        \\
    \twentyitem
    	{Jul 2018 -}
		{Sep 2018}
        {DNumeric}
        {\href{https://code.dlang.org/packages/dnumeric/}{dub}}
        {D numeric library with R syntax}
        {\begin{itemize}
        \item D numeric library for linear algebra and statistical programming with R syntax.
        \item Tools: D, dub
		\end{itemize}
        }
        \\
    \twentyitem
    	{May 2018 -}
		{Jul 2018}
        {HNumeric}
        {\href{https://hackage.haskell.org/package/HNumeric}{hackage}}
        {Haskell Numeric Library with pure functionality, R \& MATLAB Syntax.}
        {\begin{itemize}
        \item Haskell numeric library implemented by pure functional paradigm.
        
        \item Tools: Haskell, Stack, Cabal, Travis CI and Hackage CI
		\end{itemize}
        }
        \\
    \twentyitem
    	{Jan 2018 -}
		{Dec 2017}
        {Hepframework}
        {\href{https://hub.docker.com/r/axect/hepframework/}{docker hub}}
        {High Energy Physics Tools on Fedora 27}
        {\begin{itemize}
        \item Docker container for High energy physics tools.
        
        \item Tools: Docker, Jupyter, Shell script
        
        \item HEP Tools: CERN-ROOT 6, MG5\_aMC, CalcHEP, MicrOmegas, OptiMass
 
		\end{itemize}
        }
        \\
    \twentyitem
    	{May 2017 -}
		{Nov 2017}
        {Yonsei HEP-COSMO Web page}
        {\href{http://nexus.yonsei.ac.kr/}{Yonsei HEP-COSMO}}
        {}
        {\begin{itemize}
        \item Laboratory web page with Django \& Linux server
        
        \item Tools: Django, Apache, MySQL
		\end{itemize}
        }
        \\
    \twentyitem
    	{Aug 2017 -}
		{Oct 2017}
        {RGE}
        {\href{https://github.com/Axect/RGE}{github}}
        {Go \& Julia package to solve Renormalization Group Equation}
        {\begin{itemize}
        \item Golang Package which solves RGE \& generates plot with Julia.
        
        \item Tools: Go, Julia, Glide
		\end{itemize}
        }
\end{twenty}

\end{document} 
